\newpage

% Из текста короткого задания на ВКР:
% 1. Определение средств разработки и архитектуры системы: разработка ее структуры;
% 2. определение набора необходимого оборудования, программного обеспечения.

% 3. Проектирование и определение спецификаций аппаратной и программной компонентов системы.
% 4. Реализация компонентов системы с использованием выбранных средств разработки.
% 5. Сборка макета аппаратной подсистемы.
% 6. Сборка, установка и тестирование программного обеспечения аппаратной и программных подсистем.
% 7. Тестирования системы, интеграции программной и аппаратных частей.

\section{Разработка программно-аппаратной системы}

В данном разделе описывается проектирование аппаратной части системы~--- считывателя бесконтактных платежных карт~--- и разработка программной части СБО~--- программного обеспечения для считывателя и мобильного приложение бесконтактной оплаты.
Для этого в соответствии с техническим заданием (ТЗ) решаются следующие задачи:

\begin{itemize}
    \item определение средств разработки и архитектуры системы: разработка ее структуры; определение набора необходимого оборудования, программного обеспечения;
    \item проектирование компонентов и определение спецификаций аппаратной части СБО, сборка макета аппаратной части системы;
    \item выбор архитектуры и подхода разработки программной подсистемы, проектирование программных компонентов СБО (программы мобильного терминала бесконтактной оплаты и мобильного приложения оплаты) и определение спецификаций компонентов;
    \item реализация компонентов системы с использованием выбранных средств разработки;
    \item сборка, установка и тестирование программного обеспечения аппаратной подсистемы и программного обеспечения программной подсистемы.
\end{itemize}

Разрабатываемая система является mPOS-терминалом с поддержкой бесконтактной оплаты.
Как любой mPOS-терминал она состоит из мобильного устройства и считывателя платежных карт, подключаемого к ней.
В соответствии с требованиями ТЗ считыватель поддерживает только бесконтактные карты и подключается беспроводным образов посредством технологии Bluetooth.
Считыватель взаимодействует с платежным средством посредством технологии NFC.

Процесс оплаты с использованием устройства представляет следующую последовательность действия:
\begin{enumerate}
    \item представить торгово-сервисного предприятия (ТСП) вводит данные о платеже в мобильное приложение и активирует считыватель;
    \item держатель карты прикладывает к считывателю средства платежа (карта или мобильное приложение, эмулирующее платежную карту);
    \item считыватель взаимодействует со средством платежа и передает данные, необходимые для формирования платежа, на мобильное устройство;
    \item мобильное устройство отправляет HTTP-запрос на сервер банка-эквайера для выполнения платежа;
    \item сервер банка-эквайера отправляет HTTP-ответ на мобильное устройство, сообщая о статусе операции.
\end{enumerate}

Плательщик (держатель карты) может убрать средство платежа от считывателя после того как считыватель выполнит все необходимые операции для получения данных, необходимых для формирования платежа, либо сообщит о невозможности продолжения транзакции из-за ошибок совместимости или выполнения взаимодействия.


\subsection{Разработка структуры системы}

Структура разрабатываемой системы представлена на рисунке~\ref{fig:struct_scheme}.
Направления стрелок показывают направления обмена данными между программными и аппаратными модулями.

\begin{figure}[H]
    \centering
    \includegraphics[width=0.8\textwidth]{images/design/struct_scheme}
    \caption{\centering Структурная схема системы}
    \label{fig:struct_scheme}
\end{figure}

Считыватель бесконтактных платежных карт включает в себя 3 аппаратных модуля в соответствии с требованиями ТЗ:
\begin{itemize}
    \item Bluetooth-модуль для связи с мобильным устройством, которое осуществляет выполнения платежных операций;
    \item NFC-модуль для взаимодействия с бесконтактной средством платежа;
    \item микроконтроллер для осуществления управления устройством (контроля работы NFC-модуля и Bluetooth-модуля).
\end{itemize}

Для микроконтроллера реализуется ПО, с помощью которого осуществляется управления NFC и Bluetooth модулями, через предоставляемые модулями интерфейсы для управления, описанных в их документации (форматом управляющих команд).
NFC-модуль подключается посредством разъема SPI, Bluetooth-модуль подключается посредством разъема USART для передачи управляющих команд с необходимыми данными.


Мобильное приложение включает в себя следующие программные модули:
\begin{itemize}
    \item модуль взаимодействия с устройством считывателем по Bluetooth,
    \item модуль платежных операций для сетевых запросов к API банка-эквайера.
\end{itemize}

Используются следующие аппаратные модули мобильного устройства:
\begin{itemize}
    \item Bluetooth-модуль для взаимодействия со считывателем,
    \item встроенные сетевые модули (Wi-Fi, GSM, 3G, 4G) для взаимодействия с сервером банка-эквайера,
    \item дисплей для взаимодействия с пользователем.
\end{itemize}

Структурная схема с внешними участниками процесса оплаты представлена на рисунке~\ref{fig:struct_scheme_out}.
На ней также указаны направления обмена данными между программными, аппаратными модулями и внешними участниками процесса оплаты.

\begin{figure}[H]
    \centering
    \includegraphics[width=0.8\textwidth]{images/design/struct_scheme_out}
    \caption{\centering Структурная схема системы с внешними участниками процесса оплаты}
    \label{fig:struct_scheme_out}
\end{figure}


\subsection{Разработка аппаратной части системы}

\subsubsection{Функциональная схема аппаратной части системы}

В соответствии с требованиями технического задания основные функции аппаратной части системы (микроконтроллерной системы):
\begin{itemize}
    \item взаимодействие со средством платежа посредством технологии NFC в соответствии со стандартом взаимодействия с бесконтактными средсвами платежа EMV Contactless, рассмотренном в исследовательской части;
    \item связь с мобильным устройством посредством технологии Bluetooth (получение управляющих команд и отправка полученных данных).
\end{itemize}

В данном процессе задействуется МК, Bluetooth-модуль и NFC-модуль.
Все эти компоненты работают как единая система под управлением МК.

% TODO нарисовать схему работы (диаграмму последовательностей)
Алгоритм работы устройства:
\begin{enumerate}
    \item настройка Bluetooth-модуля для подключения к мобильному устройству;
    \item установка соединения мобильного устройства и Bluetooth-модуля;
    \item получение команды от мобильного устройства о начале транзакции и суммы транзакции;
    \item включение и базовая настройка NFC-модуля микроконтроллером, начало поиска средства платежа;
    \item обнаружение средства платежа, установка соединения и взаимодействие с ним в соответствии алгоритмом представленном на рисунке~\ref{fig:kernel_transaction_flow}; % TODO: нарисовать mir_transaction и заменить
    \item передача полученных данных от средства платежа на мобильное устройство посредством Bluetooth;
\end{enumerate}

Структурная схема устройства представлена на рисунке~\ref{fig:apparat_struct}.
На ней NFC-модуль имеет детализированное изображение с целью демонстрации комплексности структуры данного устройства.

\begin{figure}[H]
    \centering
    \includegraphics[width=0.8\textwidth]{images/design/apparat_struct}
    \caption{\centering Структурная схема считывателя бесконтактных платежных карт}
    \label{fig:apparat_struct}
\end{figure}


Согласно требованию ТЗ разрабатываемое устройство должно обеспечивать возможность подключения по Bluetooth к мобильному устройству, с возможностью приема и передачи данных, также оно должно взаимодействовать с бесконтактным средством платежа в соответствии со стандартом ISO/IEC 14443 с помощью модуля NFC, поддерживающему работу с картами ПС <<МИР>>.
Данные модули должны работать под управлением МК, имеющего разъемы SPI и USART для их подключения.
При этом стандарта EMV Contactless вводит строгие временные ограничения на время прямого непосредственного взаимодействия с картой (не более 0.4 мс).
Из чего следует, что МК должен обладать высокой производительностью.
ISO/IEC 14443-A используемый в платежных картах обладает скоростью передачи данных 106 кбит/с или 13.25 кбайт/с, что с учетом используемого компактного формата TLV для передаваемых сообщений и их размера, который значительно меньше чем объем, предаваемый даже за 1 мс, не создает ограничений по производительности.

При анализе существующих mPOS-терминалов было установлено, что они используют микропроцессоры с архитектурой Advanced RISC Machines (ARM), в частности ARM 7, ARM 9, ARM 11, которые обладают частотой более 100 МГц.
С учетом того, что данные терминалы используются для обработки транзакций различных платежных систем и не только бесконтактным образом для разрабатываемой системы данные процессоры будут избыточны, однако использование ARM архитектуры было бы крайне желательно.

В таблице~\ref{tab:microcontroller_comparison} приведены характеристики различных микроконтроллеров от разных компаний~\cite{stm32f103_datasheet}\cite{atmega328p_datasheet}.

\begin{longtable}[l]{|
P{0.18\textwidth}|
P{0.2\textwidth}|
P{0.15\textwidth}|
P{0.13\textwidth}|
P{0.2\textwidth}|}

    \caption{Сравнение характеристик часто используемых микроконтроллеров}
    \label{tab:microcontroller_comparison} \\
    \hline
    \textbf{Характеристика} &
    \textbf{ATmega328P} &
    \textbf{ESP8266} &
    \textbf{PIC16F8} &
    \textbf{STM32F103} \\
    \hline
    \endfirsthead

    \caption*{Продолжение таблицы~\ref{tab:microcontroller_comparison}} \\
    \hline
    \textbf{Характеристика} &
    \textbf{ATmega328P} &
    \textbf{ESP8266} &
    \textbf{PIC16F8} &
    \textbf{STM32F103} \\
    \hline
    \endhead

    \hline
    \endfoot

    \hline
    \endlastfoot

    Разрядность процессора &
    8 бит &
    32 бит &
    8 бит &
    32 бит \\
    \hline

    Тактовая частота &
    до 20 МГц &
    до 160 МГц &
    до 20 МГц &
    до 72 МГц \\
    \hline

    Флэш-память &
    32 КБ &
    4 МБ &
    14 КБ &
    64 КБ \\
    \hline

    SRAM &
    2 КБ &
    160 КБ &
    368 Б &
    20 КБ \\
    \hline

    EEPROM &
    1 КБ &
    Нет &
    256 Б &
    Нет \\
    \hline

    Количество I/O портов &
    23 &
    11 &
    13 &
    37 \\
    \hline

    Поддержка Wi-Fi &
    Нет &
    Да &
    Нет &
    Нет \\
    \hline

    Поддержка Bluetooth &
    Нет &
    Нет &
    Нет &
    Нет \\
    \hline

    Энергопотребление &
    Низкое &
    Среднее &
    Среднее &
    Низкое \\
    \hline

    Стоимость &
    Низкая &
    Низкая &
    Низкая &
    Низкая \\
    \hline
\end{longtable}

В ходе выполнения курсовой работы по дисциплине <<Микропроцессорные системы>> выбор был сделан в пользу микроконтроллера ATmega328P в составе платы Arduino Uno R3.
Данный вариант был удобен при первоначальном макетировании системы и изучении стандартов взаимодействия бесконтактной карты.
Однако частота данного МК значительно ниже чем у аналогов, у него 8-битная гарвардская архитектура процессора, при прочих равных Flash-памяти, SRAM, количество I/O-портов меньше чем у STM32F103.
Поэтому в качестве МК для аппаратной части системы используется STM32F103C8T6 на базе микропроцессора ARM Cortex-M3 с 32-битной архитектурой.
Данная архитектура позволяет работать с 32-битными регистрами, выполнять сложные математические операции, включая работу с плавающей запятой (с помощью программных или аппаратных средств).
Поддержка современных продвинутых программных библиотек, таких как HAL (Hardware Abstraction Layer), LL (Low Layer) и CMSIS (Cortex Microcontroller Software Interface Standard), предоставляет разработчику полный контроль над аппаратной частью микроконтроллера, с пониманием происходящих процессов на уровне регистров и таймеров.
В то же время, использование ATmega328P с Arduino.h для аналогичных задач часто приводит к неэффективному использованию ресурсов, ограничению функциональности и снижению производительности.
Кроме того, экосистема STM32 предоставляет мощные инструменты разработки, такие как STM32CubeIDE, которые упрощают настройку периферии и генерацию кода, сохраняя при этом гибкость и контроль над аппаратными ресурсами.

Для реализации беспроводного Bluetooth-соединения был выбран Bluetooth-модуль HC-05, обладающий следующими характеристиками:

\begin{itemize}
    \item напряжение питания: 3,3 В – 5 В;
    \item потребляемый ток: при подключении – до 40 мА (поиск, сопряжение, подключение), при передаче данных – до 8 мА;
    \item частотный диапазон: 2,4 ГГц – 2,48 ГГц;
    \item мощность передатчика: до +4 дБм;
    \item чувствительность приемника: 80 дБм;
    \item дальность связи: до 10 метров;
    \item интерфейс: UART (с последовательной передачей данных);
    \item поддерживаемые скорости передачи данных: 9600, 19200, 38400, 57600, 115200, 230400 и 460800 бит/сек;
    \item режимы работы: Master (ведущий) и Slave (ведомый);
    \item рабочая температура: от -25 °C до +75 °C;
    \item размеры: 27 x 13 x 2,2 мм~\cite{hc05_datasheet}.
\end{itemize}

HC-05 имеет дальность связи до 10 метров, поддерживает скорости передачи данных до 460800 бит/сек и может работать как в режиме Master (ведущий), так и в режиме Slave (ведомый).
Единственным недостатком является не самая актуальная версия технологии Bluetooth 2.0.
Однако даже модуль HC-08 имеет версию 4.0 (актуальная~-- 6.0) при стоимости в 4--5 раз выше.
Т.к. осуществляется процедура передачи данных без необходимости частого и многоразового подключения устройства-хоста к bluetooth-модулю, а приоритетом является низкое энергопотребление – стандарт Bluetooth 2.0 является преимуществом, а не недостатком.

Модуль HC-05 можно настроить с помощью AT-команд, которые отправляются через интерфейс USART.
Это позволяет изменить имя устройства, пароль для подключения, скорость передачи данных и другие параметры.
Примеры AT-команд:

\begin{itemize}
    \item AT – проверка связи с модулем;
    \item AT+NAME=NewName – изменение имени модуля;
    \item AT+UART=9600 – установка скорости передачи данных на 9600 бод.
\end{itemize}

Модуль HC-05 включает в себя чип BC417143 и работает на напряжении 5 В и реализуя прием и передачу сигнала.


Среди прочих NFC-модулей наиболее популярными являются модули компании NXP.
К тому они имеют качественную и исчерпывающую документацию.
NXP поставляет не только NFC-модули, но и микроконтроллеры с интегрированными модулями NFC, а также разрабатывает ПО для работы с модулями, однако только для МК собственного производства.

В качестве модуля для считывателя используется NXP PN5180.
Данный модуль имеет поддержку большого числа стандартов технологии NFC.
Имеет сравнительно низкую стоимость (порядка 700--1000 рублей на ноябрь 2024 года).
А также имеет широкий функционал и подробную документацию.
PN5180 является широко распространенным модулем, с поддержкой множества стандартов (в том числе и используемого в платежах~-- ISO14443-A).
Альтернативой ему могут выступать схожие модули от компании NXP, в частности PN5190 (улучшенная версия PN5180), однако он умеет выше стоимость и при этом улучшения в производительности и поддержки стандартов пренебрежимо малы в рамках данного устройства.

%Основные внутренние компоненты интегральной схемы модуля представлены на рисунке~\ref{fig:pn5180_components}~\cite{pn5180_datasheet}.
%
%\begin{figure}[H]
%    \centering
%    \includegraphics[width=0.8\textwidth]{images/design/pn5180_components}
%    \caption{\centering Компоненты интегральной схемы модуля PN5180}
%    \label{fig:pn5180_components}
%\end{figure}

Важно отметить, что производитель модуля предварительно выполняет следующие действия по его настройке:
\begin{itemize}
    \item определение целевого импеданса для оптимизации мощности RF-выхода и минимизации потребление энергии;
    \item проектирование EMC-фильтра для подавления нежелательных гармоник тока, которые могут создавать помехи в работе устройств;
    \item измерение LCR-параметров антенны (индуктивности, емкости и сопротивления) для работы;
    \item расчет компонентов для согласования антенны с NFC-модулем.
\end{itemize}

Также производится симуляция работы антенны, тестирование работы в реальных условиях и корректировка РЧ контура.
В результате чего получается надежный и стабильно работающий модуль, для которого нет необходимости в выполнении данных затруднительных действий, требующих дополнительного оборудования и навыков.


PN5180 использует для связи с микроконтроллером SPI, расширенный линией BUSY, что позволяет ему указыать о невозможности приема и отправки данных в конкретный момент времени.
Максимальная скорость его разъема SPI составляет 7 Мбит/с, при этом параметры SPI устанавливаются следующим образом: CPOL=0, CPHA=0 (SPI Mode 0), старший бит первым (MSB first).
На модуль по SPI передаются управляющие команды, список которых приведен в его datasheet в разделе <<Работа модуля NFC>>, с помощью этих команд происходит настройка модуля\cite{pn5180_datasheet}.

Уровень напряжения сигналов, которыми оперирует модуль, составляет 1,8--3,3 В.
У STM32F103C8T6 рабочий уровень напряжения на разъемах~--- 3,3 В, что делает его совместимым с данным модулем.
В свою очередь на разъемах микроконтроллера ATmega328P рабочим уровнем напряжения является 5 В, что усложняло его использование в первом разработанном макете устройства (возникала необходимость использования конвертера уровня напряжения, что повышало потрбляемую МК-системой мощность).


На функциональной схеме изображены используемые компоненты, необходимые для работы устройства.
Она показывает логическое соединение устройств, используемые для подключения контакты, а также направление передачи данных и управляющих сигналов между компонентами.
Спроектированная функциональная схема разработанного устройства представлена на рисунке~\ref{fig:func_scheme}, а также в приложении В.
У МК изображены только используемые компоненты.

\begin{figure}[H]
    \centering
    \includegraphics[width=1\textwidth]{images/design/func_scheme}
    \caption{\centering Функциональная схема считывателя бесконтактных платежных карт}
    \label{fig:func_scheme}
\end{figure}





\subsubsection{Принципиальная схема аппаратной части системы}

%\paragraph{Программирование МК}

Программирование микроконтроллера STM32F103C8T6 осуществляется с использованием интерфейса SWD (Serial Wire Debug), который является стандартным решением для отладки и прошивки большинства 32-битных ARM-микроконтроллеров.
Интерфейс SWD разработан для упрощенного подключения к микроконтроллеру, в отличие от JTAG, он использует минимальное количество выводов, что делает его особенно удобным в условиях ограниченного пространства на плате.

Для работы с этим интерфейсом необходимы всего два сигнальных контакта:
\begin{itemize}
    \item SWCLK — тактовая линия;
    \item SWDIO — двунаправленная линия передачи данных.
\end{itemize}

Также требуются линии питания и земли:
\begin{itemize}
    \item VCC — питание 3.3 В;
    \item GND — общий провод.
\end{itemize}

Процесс программирования выглядит следующим образом:

\begin{enumerate}
    \item программатор ST-Link v2 подключается к микроконтроллеру через разъем, соответствующий интерфейсу SWD;
    \item программатор устанавливает связь с ядром микроконтроллера, отправляя тактовые сигналы по линии SWCLK и управляя данными через SWDIO;
    \item происходит загрузка и запись кода во флэш-память микроконтроллера;
    \item МК автоматически перезапускается для исполнения загруженную программы.
\end{enumerate}

Таким образом, для программирования микроконтроллера STM32F103C8T6 через ST-Link v2 используется разъем SWD, представленный на рисунке~\ref{fig:swd}.

\begin{figure}[H]
    \centering
    \includegraphics[width=0.4\textwidth]{images/design/swd}
    \caption{\centering Разъем программирования МК}
    \label{fig:swd}
\end{figure}

%\paragraph{Подключение цепи питания}

Для обеспечения работоспособности устройства необходимо подавать стабилизированное напряжение на микроконтроллер и прочие компоненты схемы.
В данной МК-систем питание микроконтроллера STM32F103C8T6 осуществляется через стандартный USB-разъём microUSB, который используется как интерфейс подключения к внешнему источнику питания.
Это позволяет отказаться от использования дополнительного внешнего источника питания и упрощает конструкцию устройства.
Разъем представлен на рисунке~\ref{fig:usb}.

\begin{figure}[H]
    \centering
    \includegraphics[width=0.4\textwidth]{images/design/usb}
    \caption{\centering Разъем питания МК}
    \label{fig:usb}
\end{figure}

Напряжение с USB поступает непосредственно на стабилизатор напряжения RT9193--33, т.к. микроконтроллер STM32F103C8T6 работает на напряжении 3.3 В, а не 5 В, как это часто бывает в других МК.
Стабилизатор преобразует напряжение с USB (5 В) к необходимому уровню и обеспечивает стабильное питание ядра и периферийных модулей микросхемы.

Для фильтрации питающего напряжения используются керамических конденсаторы, установленные на входе и выходе стабилизатора.
Такое решение обеспечивает минимальный уровень шума и стабильную работу микроконтроллера.
Емкости конденсаторов определены в соответствии с документацией STM32F103C8T6~\cite{stm32f103_datasheet}.

Схема стабилизации и понижения напряжения представлена на рисунке~\ref{fig:mk_usb_stab}.

\begin{figure}[H]
    \centering
    \includegraphics[width=0.4\textwidth]{images/design/mk_usb_stab}
    \caption{\centering Схема стабилизации питания МК-системы}
    \label{fig:mk_usb_stab}
\end{figure}

%\paragraph{Расчет сопротивления резисторов}

На принципиальной схеме расположено несколько резисторов.

Резистор R1 используется для стабилизации тока на источнике тактирования.
Резистор R2 используется для подтягивания линии RESET до 3.3 В.
Их номиналы установлены согласно документации на микроконтроллер\cite{stm32f103_datasheet}.

Резисторы R2--R5 добавлены для токоорганичния разъема USB, который используется для питания.

В Bluetooth-модуле, согласно документации, Резисторы R9, R6 имеют номинал 1 кОм и 220 кОм соответственно, а R7, R8~--- 10 кОм.

В состав NFC-модуля PN5180, согласно документации, входят резисторы R10, R13 сопротивлением 4.7 кОм и R11, R12, R14, R15 сопротивлением 2.2 кОм.

Дополнительно хотелось бы отметить, что расчет сопротивлений, емкостей и индуктивностей для PN5180 производился на основе раздела datasheet <<17.1 Typical component values>>\cite{pn5180_datasheet}.
В нем приведена схема с минимальным количеством компонентов и значения для них, представленные на рисунках~\ref{fig:pn5180mincomp} и~\ref{fig:pn5180mincompvalues} соответственно.


\begin{figure}[H]
    \centering
    \includegraphics[width=0.7\textwidth]{images/design/pn5180mincomp}
    \caption{\centering Схема модуля PN5180 с минимальным набором элементов}
    \label{fig:pn5180mincomp}
\end{figure}


\begin{figure}[H]
    \centering
    \includegraphics[width=0.7\textwidth]{images/design/pn5180mincompvalues}
    \caption{\centering Значения для элементов PN5180}
    \label{fig:pn5180mincompvalues}
\end{figure}

Также в открытом доступе находится принципиальная электрическая схема первой ревизии модуля PN5180 от производителя, которая и послужила основой для собственной электрической принципиальной схемы\cite{pn5180_schematic}.

% TODO: Подключение модулей

На основе всех вышеописанных сведений была спроектирована принципиальная схема разрабатываемой системы, показанная на рисунке~\ref{fig:principal_scheme} и в приложении Г.

\begin{figure}[H]
    \centering
    \fbox{\includegraphics[angle=90, height=0.85\textheight]{appendices/schemas_princip.pdf}}
    \caption{\centering Схема электрическая принципиальная разрабатываемого устройства}
    \label{fig:principal_scheme}
\end{figure}





\subsubsection{Макет аппаратной части системы}

В качестве МК для макета используется STM32F103C8T6 в составе отладочной платы Blue Pill, представленной на рисунке~\ref{fig:blue_pill}.

\begin{figure}[H]
    \centering
    \includegraphics[width=0.8\textwidth]{images/design/blue_pill}
    \caption{\centering Отладочная плата Blue Pill с МК STM32F103C8T6}
    \label{fig:blue_pill}
\end{figure}

Сборка макета происходила в соответствии с принципиальной схемой с рисунка~\ref{fig:principal_scheme} и представлена на рисунке~\ref{fig:maket}.

\begin{figure}[H]
    \centering
    \includegraphics[width=0.5\textwidth]{images/design/maket}
    \caption{\centering Собранный макет устройства}
    \label{fig:maket}
\end{figure}

Для разработанного устройства-считывателя также был произведен расчет мощности.
Мощность, потребляемая разработанной схемой, складывается из мощностей, потребляемых всеми устройствами из ее состава.
Из документации используемых устройств получена информация, приведенная в таблице~\ref{tab:power}.
Расчет максимальной мощности, потребляемой устройством, определяется по формуле:

$$
P_{max} = \sum_{i=1}^{k} U_i \cdot I_{i_{max}} \cdot n_i
$$
где $U_i$~--- напряжение питания устройства, $I_{i_{max}}$~--- максимальный ток потребления, $n_i$~--- количество элементов в системе.

\begin{longtable}[l]{|
P{0.23\textwidth}|
P{0.15\textwidth}|
P{0.15\textwidth}|
P{0.15\textwidth}|
P{0.2\textwidth}|}

    \caption{Сравнение микросхем по потребляемой мощности}
    \label{tab:power} \\
    \hline
    \textbf{Микросхема} &
    \textbf{Ток потребления, мА} &
    \textbf{Потребляемая мощность, мВт} &
    \textbf{Количество устройств} &
    \textbf{Суммарная мощность, мВт} \\
    \hline
    \endfirsthead

    \caption*{Продолжение таблицы~\ref{tab:power}} \\
    \hline
    \textbf{Микросхема} &
    \textbf{Ток потребления, мА} &
    \textbf{Потребляемая мощность, мВт} &
    \textbf{Количество устройств} &
    \textbf{Суммарная мощность, мВт} \\
    \hline
    \endhead

    \hline
    \endfoot

    \hline
    \endlastfoot

    STM32F103\-C8T6 & 10 & 50 & 1 & 50 \\ \hline

    BC417143 & 70 & 231 & 1 & 231 \\ \hline

    PN5180A0NH & 20 & 60,6 & 1 & 60,6 \\ \hline

    R1114-3.3 & 100 & 500 & 1 & 500 \\ \hline

    CD1206-S01575 & 150 & 750 & 1 & 750 \\ \hline

    LP2985-33DBVR & 150 & 750 & 1 & 750 \\ \hline

    NCP1117-ST50T3G & 800 & 4500 & 1 & 4500 \\ \hline
\end{longtable}

При питании 5 В суммарная мощность системы будет приблизительно равна 6,342 Вт.

% TODO: make correct расчет потребляемой мощности


\subsection{Разработка программной части системы}

\subsubsection{Программа считывателя платежных средств}

% todo: добавить рисунок для mir

В основу алгоритмов работы устройства легли алгоритмы, представлены на рисунках~\ref{fig:pcd_flow} и~\ref{fig:pcd_flow_2_picc_activation},~\ref{fig:transaction_flow_example} и~\ref{fig:kernel_transaction_flow}, а также алгоритмы из спецификации для ПС <<МИР>>~\cite{book_mir}.
На основе них были разработаны несколько алгоритмов:

\begin{itemize}
    \item алгоритм перезагрузки NFC-модуля, представленный на рисунке~\ref{fig:restart_pn5180};
    \item алгоритм выполнения платежной транзакции, представленный на рисунке~\ref{fig:complete_transaction};
    \item алгоритм активации протокола ISO/IEC 14443--3, представленный на рисунке~\ref{fig:activate_iso3};
    \item алгоритм активации протокола ISO/IEC 14443--4, представленный на рисунке~\ref{fig:activate_iso4};
    \item алгоритм поиска платежных данных AFL, представленный на рисунке~\ref{fig:find_afl}.
\end{itemize}


\begin{figure}[H]
    \centering
    \includegraphics[width=0.9\textwidth]{images/design/complete_transaction}
    \caption{\centering Схема алгоритма выполнения платежной транзакции}
    \label{fig:complete_transaction}
\end{figure}

\begin{figure}[H]
    \centering
    \includegraphics[width=0.8\textwidth]{images/design/activate_iso3}
    \caption{\centering Схема алгоритма активации протокола ISO/IEC 14443--3}
    \label{fig:activate_iso3}
\end{figure}

\begin{figure}[H]
    \centering
    \includegraphics[width=0.5\textwidth]{images/design/activate_iso4}
    \caption{\centering Схема алгоритма активации протокола ISO/IEC 14443--4}
    \label{fig:activate_iso4}
\end{figure}

\begin{figure}[H]
    \centering
    \includegraphics[width=0.3\textwidth]{images/design/restart_pn5180}
    \caption{\centering Схема алгоритма перезагрузки NFC-модуля}
    \label{fig:restart_pn5180}
\end{figure}

\begin{figure}[H]
    \centering
    \includegraphics[width=0.6\textwidth]{images/design/find_afl}
    \caption{\centering Схема алгоритма поиска платежных данных AFL}
    \label{fig:find_afl}
\end{figure}


На рисунке~\ref{fig:find_afl} представлен алгоритм поиска значения тега Application File Locator, а точнее того, чтение которого является обязательным для активации транзакции, на это указывает numRecords != 0, если оно не равно 0, то эту запись нужно считать для транзакции (пояснение приведено на рисунке~\ref{fig:afl_pic}), в то время как для других записей в AFL данный тег установлен не будет.

\begin{figure}[H]
    \centering
    \includegraphics[width=0.7\textwidth]{images/design/afl_pic}
    \caption{\centering Формат кадра AFL}
    \label{fig:afl_pic}
\end{figure}

На основе найденного значения формируется запрос Read Record.
Правила формирования следующие:

\begin{itemize}
    \item CLA = 0x00  - класс команды,
    \item INS = 0xB2 - тип инструкции (Read Record),
    \item P1 = Start - индекс первого байта,
    \item P2 = SFI|4 - второй параметр команды на основе SFI,
    \item Lc = 0x00 - ожидаемое количество байт в ответе (0 - любое).
\end{itemize}

Разработка программного обеспечения устройства происходила в среде STM32CubeIDE, предназначенной для разработки на микроконтроллерах серии STM32.
STM32 умеет производить компиляцию с файлов C и C++ с помощью gcc и g++.
В качестве основного языка программирования использовался C++, т.к. в отличие от C он предоставляет широкую поддержку парадигмы объектно-ориентированного программирования.
Сама программа имеет модульный характер, на основе того, что в соответствие каждому аппаратному модулю (кроме МК) поставлен программный модуль, реализующий основные свойства и методы класса.

Настройка STM32F103C8T6 производилась с помощью графических средств STM32CubeIDE. Пример такой настройки является настройка источника тактирования, представленная на рисунке~\ref{fig:stm_cube}.

\begin{figure}[H]
    \centering
    \includegraphics[width=0.9\textwidth]{images/design/stm_cube}
    \caption{\centering Настройка источника тактирования STM32F103C8T6}
    \label{fig:stm_cube}
\end{figure}

В качестве частоты используется 14 МГц для достижения максимальной скорости передачи данных с PN5180, который поддерживает максимальную скорость в 7 Мбит/с.
С помощью предделителя частоты SPI, равного 2, частота APB2 (именно к ней подключен SPI1) понижается до 7 МГц и 7 Мбит/с соответственно.
В приложении Б приведены листинги со следующими фрагментами кода:
\begin{itemize}
    \item настройка SPI для подключения PN51804
    \item настройка USART для подключения Bluetooth-модуля HC-05.
\end{itemize}

Взаимодействие с PN5180 разделено на два класса PN5180 и PN5180ISO14443.
Первый описывает реализацию основных методов управления состоянием модуля, а второй реализует специфичные для стандарта ISO14443 методы работы, с помощью которых осуществляется взаимодействие NFC-модуля и платежного средства.
Примеры процедур, реализующих их, также приведены в приложении Б.


\subsubsection{Программа мобильного устройства}

В соответствии с требованиями ТЗ прогрмма должна работать под управлением устройств с ОС Android версии 9 и выше.
Разрабатываемое программное обеспечение для мобильного устройства должно обеспечивать выполнение всех требуемых функций взаимодействия между считывателем бесконтактных карт и сервером банка-эквайера.
Управляющая программа будет работать на мобильном устройстве под управлением ОС Android и предоставлять пользовательский интерфейс для отслеживания состояния транзакции, её инициализации и получения результата.

Основные функции, выполняемые программой:
\begin{itemize}
    \item инициация взаимодействия между платежным средством (банковской картой или мобильным кошельком) и терминалом;
    \item взаимодействие с внешним сервером через протоколы HTTPS и REST API, с соблюдением требований стандарта PCI DSS ;
    \item управление состоянием транзакции на основе данных, полученных от сервера и терминала, в соответствии с правилами EMV Transaction Processing и форматом сообщений ISO 8583 ;
    \item корректное завершение платежной операции, включая отображение результатов пользователю и передачу статуса транзакции.
\end{itemize}

Для начала транзакции программа использует данные, полученные от платежного терминала и карты.
Конкретный перечень полей определяется спецификой взаимодействия с каждой из поддерживаемых платежных систем.
При этом данные должны быть корректно интерпретированы и переданы на сервер эквайера через безопасное соединение.

Программа возвращает статус выполнения платежной операции в формате JSON, совместимом с REST API.
Пример такого ответа может быть следующее:
{
    <<status>>: <<success>>,
    <<transaction\_id>>: <<1234567890>>,
    <<amount>>: 1500.00,
    <<timestamp>>: <<2025-04-05T14:30:00Z>>
}

При этом максимальное время ожидания ответа от аппаратной части системы установлено на уровне 30 секунд, от внешнего сервера~--- 10 секунд.


Для реализации мобильного приложения была выбрана платформа Android SDK, так как она:
\begin{itemize}
    \item предоставляет полный доступ к низкоуровневым модулям, таким как Bluetooth и NFC;
    \item поддерживает современные протоколы шифрования и интеграции с REST API;
    \item обеспечивает гибкость в версионировании и обновлениях;
    \item использует Kotlin или Java, что позволяет писать чистый и понятный код, сохраняя высокую степень переносимости и производительности.
\end{itemize}

Для реализации сетевого взаимодействия используется Retrofit 2 и OkHttp, что обеспечивает:

\begin{itemize}
    \item Удобную работу с REST API;
    \item Поддержку TLS 1.2+;
    \item Возможность добавления сложных заголовков, таких как авторизация и проверка целостности запросов.
\end{itemize}

Для тестирования и отладки применяются:

\begin{itemize}
    \item OkHttp Profiler — для контроля сетевых запросов;
    \item Mockk / JUnit — для автоматизированного тестирования бизнес-логики без необходимости использования реального терминала.
\end{itemize}

% TODO: диаграммы деятельности и чего-либо еще

На основе процесса выполнения транзакции, описанного в спецификации EMV и определенности действий, выполняемых устройством-считывателем, были определены основные сценарии использования приложения, также был сформирован список необходимых экранов:
\begin{itemize}
    \item авторизация,
    \item выбор устройства,
    \item создание платежа,
    \item ожидание инициализации карты,
    \item успех/ошибка при инициализации карты,
    \item ввод PIN-кода,
    \item ввод подписи,
    \item успех/ошибка при выполнении платежа.
\end{itemize}

Формы всех экранов представлены на риснуке~\ref{fig:screens}.

\begin{figure}[H]
    \centering
    \includegraphics[width=0.9\textwidth]{images/design/screens}
    \caption{\centering Формы интерфейсы экранов}
    \label{fig:screens}
\end{figure}

Экран авторизации должен иметь поля для ввода логина и пароля, и выполнять аутентификацию и авторизацию пользователя приложения с помощью введенных логина и пароля по нажатию кнопки <<Войти>>.
В случае успешной авторизации выполняется переход на экран выбора устройства для подключения.
В случае ошибки отображается сообщение об ошибке в нижней части экрана, поля ввода логина и пароля очищаются, пользователь может повторно ввести их и повторить попытку авторизации.

На экране выбора устройства отображаются все доступные для подключения Bluetooth-устройства.
При первом открытие экрана выполняется проверка наличия разрешений на подключение и обнаружение устройств поблизости посредством Bluetooth.
По нажатию на устройство должно выполняться сопряжение и/или подключение к выбранному устройству.
Если выбранное устройство~--- считыватель бесконтактных карт, то он сигнализирует (передает данные) об этом после подключения.
В случае успешного подключения к считывателю выполняется переход на экран создания платежа.
В случае неудачного подключения отображается сообщение об ошибке в нижней части экрана, пользователь может повторить попытку подключения к считывателю.

На экране создания платежа пользователь находится только при наличии активного подключения к считывателю.
Он может ввести сумму оплаты, а также, при необходимости, номер телефона и/или электронную почту для отправки электронного чека об операции.
По нажатию на кнопку <<Настройка>> происходит переход к экрану выбору устройств.
По нажатию на кнопку <<Очистить>> происходит очистка полей ввода суммы транзакции, полей ввода номера телефона и/или электронной почты.
По нажатию на кнопку <<Оплатить>> валидируются введеные пользователем данные, в случае их корректности отображается экран ожидания инициализации карты, в случае некорректности некорректные поля подсвечиваются, ошибка выводится на экран.

При переходе на экран ожидания инициализации карты на считыватель передается команда о необходимости запуска NFC-модуля, после его активации выполняется поиск бесконтактного платежного средства и взаимодействие с ним по спецификации ПС.
Если данные процессы выполнены успешно, то приложение получает от считывателя все необходимые данные для формирования платежной транзакции и отображает экран успешной инициализации карты, в противном случае~-- экран ошибки инициализации карты, на котором есть кнопка <<Повторить попытку>>, запускающая повторную инициализацию NFC-модуля и взаимодействие с картой.

Карта посредством считывателя может запросить дополнительную проверку в виде ввода PIN-кода держателем карты, в этом случае открывается экран ввода PIN-кода, на котором держатель карты может ввести кода от своей платежной карты.
Также проверка может запросить в виде ввода подписи держателя карты.
И PIN-код, и подпись отправляются в запросе в банк-эквайер посредством REST API с целью аутентификации держателя карты, банк отвечает статусом проверки держателя карты.

После успешной инициализации и прохождении проверок (либо их отсутствия) отправляется запрос в банк-эквайер посредством REST API с суммой операции и всеми необходимыми данными для выполнения платежа.
Банк-эквайер возвращает данные о результате выполнения платежной транзакции.
В зависимости от результата либо отображается экран успешного выполнения транзакции с информацией о транзакции, либо отображается экран с описанием произошедшей ошибки.
Оба экрана позволяют по нажатию кнопки перейти на экран создания платежа и создать новую платежную транзакцию.


На основе данного описания экранов был спроектирован граф состояний интерфейса, представленная на рисунке~\ref{fig:nav}.

\begin{figure}[H]
    \centering
    \includegraphics[width=0.8\textwidth]{images/design/nav}
    \caption{\centering Граф состояний интерфейса}
    \label{fig:nav}
\end{figure}

На котором введены следующие обозначения событий:
\begin{itemize}
    \item С0~--- открытие приложения,
    \item С1~--- нажатие системной кнопки <<Назад>>,
    \item С2~--- успешная авторизация после ввода данных и нажатия кнопки <<Войти>>,
    \item С3~--- ошибка авторизации после нажатия кнопки <<Войти>>,
    \item С4~--- успешное подключение к устройству после выбора устройства и нажатия <<Продолжить>>,
    \item С5~--- ошибка подключение к устройству после выбора устройства и нажатия <<Продолжить>>
    \item С6~--- нажатие иконки <<Настройки>>,
    \item С7~--- нажатие иконки <<Очистить данные>>,
    \item С8~--- нажатие кнопки <<Оплатить>>,
    \item С9~--- нажатие кнопки <<Отменить>>,
    \item С10~--- успешная инициализация карты,
    \item С11~--- ошибка при инициализации карты,
    \item С12~--- нажатие кнопки <<Повторить попытку>>,
    \item С13~--- успешное выполнение платежа,
    \item С14~--- нажатие кнопки <<Новый платеж>>,
    \item С15~--- запрос ввода подписи держателя карты,
    \item С16~--- запрос ввода PIN-кода держателя карты,
    \item С17~--- ошибка выполнения платежа,
    \item С18~--- ошибка проверки PIN-кода,
    \item С19~--- успешная проверка PIN-кода и ошибка при выполнении платежа,
    \item C20~--- успешная проверка PIN-кода и успешное выполнение платежа,
    \item С21~--- ошибка проверки подписи,
    \item С22~--- успешная проверка подписи и ошибка при выполнении платежа,
    \item С23~--- успешная проверка подписи и успешное выполнение платежа.
\end{itemize}


Для разрабатываемой программы была спроектирована диаграмма классов, показывающая особенности интеграции бизнес-логики и интерфейса экранов.
Диаграмма представлена на риснуке~\ref{fig:classes}.
На ней изображены классы модулей авторизации, подключения устройства-считывателя, выполнения платежа.

\begin{figure}[H]
    \centering
    \includegraphics[width=0.9\textwidth]{images/design/classes}
    \caption{\centering Диаграмма классов мобильного приложения}
    \label{fig:classes}
\end{figure}

Для разработанного приложения было проведено структурное тестирование методом покрытия операторов.
Выполнения тестовых случаев и их проверка осуществлялось с помощью библиотек Mockk и JUnit для структурного и модульного тестирования.
Описание тестовых случаев, проверяющих работу устройства, перечислено в таблице~\ref{tab:mob_app_test_cases}.

\begin{longtable}[l]{| P{0.18\textwidth} | P{0.3\textwidth} | P{0.2\textwidth} | P{0.2\textwidth}|}

    \caption{Тестовые случаи структурного тестирования программной части системы}
    \label{tab:mob_app_test_cases} \\
    \hline
    \textbf{Тестируемый модуль} &
    \textbf{Описание теста} &
    \textbf{Ожидаемый результат} &
    \textbf{Полученный результат} \\
    \hline
    \endfirsthead

    \caption*{Продолжение таблицы~\ref{tab:mob_app_test_cases}} \\
    \hline
    \textbf{Тестируемый модуль} &
    \textbf{Описание теста} &
    \textbf{Ожидаемый результат} &
    \textbf{Полученный результат} \\
    \hline
    \endhead

    \hline
    \endfoot

    \hline
    \endlastfoot

    Авторизация &
    Ввод корректных логина и пароля и нажатие кнопки <<Войти>> &
    Успешная авторизация, переход на экран выбора устройства &
    Успешная авторизация, переход выполнен \\
    \hline

    Авторизация &
    Ввод некорректных данных (неверный логин или пароль) &
    Отображение ошибки, очистка полей ввода &
    Сообщение об ошибке отображено, поля очищены \\
    \hline

    Выбор устройства &
    Проверка наличия доступных Bluetooth-устройств &
    Список устройств успешно заполнен, устройства обнаружены &
    Список устройств отобразился корректно \\
    \hline

    Выбор устройства &
    Выбор считывателя бесконтактных карт из списка &
    Подключение к устройству, подтверждение типа устройства &
    Устройство определено как считыватель карт \\
    \hline

    Выбор устройства &
    Нажатие на несовместимое устройство &
    Сообщение о невозможности подключения к данному устройству &
    Пользователь уведомлен о несовместимости \\
    \hline

    Создание платежа &
    Ввод корректной суммы, номера телефона и/или email для чека, нажатие <<Оплатить>> &
    Отправка запроса на инициализацию карты &
    Данные переданы, переход на следующий экран выполнен \\
    \hline

    Создание платежа &
    Ввод некорректной суммы (например, 0 или отрицательное число) &
    Подсветка поля ввода, вывод сообщения об ошибке &
    Поле выделено, ошибка отображена \\
    \hline

    Инициализация карты &
    Поднесение бесконтактной карты к считывателю &
    Успешная инициализация карты, получение данных от неё &
    Карта распознана, данные получены \\
    \hline

    Инициализация карты &
    Поднесение несовместимого устройства или его отсутствие &
    Переход на экран ошибки инициализации карты &
    Экран ошибки отображен, есть возможность повторить попытку \\
    \hline

    Ввод PIN-кода &
    Запрос на ввод PIN-кода со стороны карты &
    Открытие экрана ввода PIN, отправка кода через REST API &
    Экран ввода отображен, данные переданы на сервер \\
    \hline

    Ввод PIN-кода &
    Ввод неверного PIN более трёх раз &
    Блокировка карты, переход на экран ошибки &
    Карта заблокирована, транзакция прервана \\
    \hline

    Ввод подписи &
    Запрос на ввод подписи держателя карты &
    Открытие экрана ввода подписи, отправка данных на сервер &
    Экран отображен, данные отправлены \\
    \hline

    Ввод подписи &
    Отказ пользователя от ввода подписи &
    Отмена транзакции по сигналу сервера &
    Транзакция отменена, переход на экран ошибки \\
    \hline

    Успех/ошибка платежа &
    Получение положительного ответа от банка &
    Отображение экрана успешной транзакции с данными о платеже &
    Экран успеха отображен, информация о платеже корректна \\
    \hline

    Успех/ошибка платежа &
    Получение отрицательного ответа от банка &
    Отображение экрана ошибки с причиной, возможность начать новую транзакцию &
    Экран ошибки отображен, можно создать новую транзакцию \\
    \hline

    Все модули &
    Обрыв связи с терминалом во время транзакции &
    Отображение ошибки связи, возможность повторного подключения &
    Ошибка связи отображена, можно повторить попытку \\
    \hline
\end{longtable}


Тестирование устройства также проводилось методом покрытия операторов.
Оно представляет собой отдельную версию программного обеспечения МК, содержащую только код тестов, расположенных в директории test исходного проекта. 
Результатом тетсирования является загрузка данного ПО на МК с выполнением всех тестов и выводом отладочных сообщений.
Значения в таблице~\ref{tab:test_cases_hardware} показывают корректность работы програмнного обеспечения МК.

\begin{longtable}[l]{| P{0.2\textwidth} | P{0.3\textwidth} | P{0.2\textwidth} | P{0.2\textwidth}|}

    \caption{Тестовые случаи структурного тестирования аппаратной части системы}
    \label{tab:test_cases_hardware} \\
    \hline
    \textbf{Тестируемый модуль} &
    \textbf{Описание теста} &
    \textbf{Ожидаемый результат} &
    \textbf{Полученный результат} \\
    \hline
    \endfirsthead

    \caption*{Продолжение таблицы~\ref{tab:test_cases_hardware}} \\
    \hline
    \textbf{Тестируемый модуль} &
    \textbf{Описание теста} &
    \textbf{Ожидаемый результат} &
    \textbf{Полученный результат} \\
    \hline
    \endhead

    \hline
    \endfoot

    \hline
    \endlastfoot

    Подключение по Bluetooth &
    Включение Bluetooth на устройстве и попытка подключения к терминалу &
    Установление устойчивого соединения между смартфоном и считывателем &
    Подключение выполнено успешно \\
    \hline

    Передача данных по Bluetooth &
    Отправка команды от приложения к терминалу через Bluetooth &
    Команда корректно передана и обработана терминалом &
    Данные получены и обработаны \\
    \hline

    Получение данных по Bluetooth &
    Прием ответа от терминала после выполнения команды &
    Приложение получает данные в корректном формате &
    Данные успешно получены и интерпретированы \\
    \hline

    Активация Bluetooth-модуля &
    Проверка наличия и активации Bluetooth-адаптера на считывателе &
    Модуль Bluetooth должен быть готов к подключению и сканированию &
    Bluetooth-модуль активирован \\
    \hline

    Активация NFC-модуля &
    Активация NFC-чипа для чтения карты &
    NFC-модуль должен быть активен и готов к работе &
    NFC-модуль успешно активирован \\
    \hline

    Получение UID карты (ISO/IEC 14443-3) &
    Поднесение бесконтактной карты к считывателю &
    Считыватель определяет идентификатор карты (UID) в соответствии с протоколом ISO/IEC 14443-3 &
    UID успешно считан и передан в приложение \\
    \hline

    Активация карты (ISO/IEC 14443-4) &
    Инициализация протокола транспортного уровня с картой &
    Установлено логическое соединение с картой, карта готова к обмену APDU-командами &
    Карта успешно активирована \\
    \hline

    Обмен APDU-командами &
    Отправка команд SELECT AID, GET PROCESSING OPTIONS &
    Карта корректно отвечает на APDU-запросы &
    APDU-обмен прошел успешно \\
    \hline

    Чтение данных с карты &
    Запрос данных о платежном приложении (например, AID, номере карты) &
    Данные успешно прочитаны и переданы в мобильное приложение &
    Данные получены корректно \\
    \hline

    Генерация криптограммы &
    Выполнение команды GENERATE AC &
    Карта генерирует Application Cryptogram, который передается в приложение &
    Криптограмма получена и передана в банк \\
    \hline

    Защита данных &
    Попытка несанкционированного доступа к данным карты через NFC-сниффинг &
    Система не должна позволять получить конфиденциальные данные без авторизации &
    Данные защищены, доступ невозможен \\
    \hline

    Работа антиколлизии &
    Поднесение нескольких карт к считывателю одновременно &
    Считыватель должен выбрать одну карту с помощью протокола антиколлизии (ISO/IEC 14443-3) &
    Одна карта успешно выбрана \\
    \hline

    Энергопотребление &
    Измерение потребляемого тока при активных Bluetooth и NFC &
    Потребление должно находиться в пределах технических требований &
    Устройство работает в допустимых пределах \\
    \hline

    Связь с сервером &
    Отправка данных с устройства на сервер эквайера через REST API &
    Данные должны быть переданы в зашифрованном виде и проверены на целостность &
    Данные переданы безопасно и корректно \\
    \hline

    Ошибки связи &
    Разрыв Bluetooth-соединения во время транзакции &
    Приложение должно уведомить пользователя, операция будет отменена &
    Связь разорвана, транзакция отменена \\
    \hline

    Обработка ошибок NFC &
    Некорректная или поврежденная карта находится в поле считывателя &
    Считыватель должен распознать ошибку и сообщить пользователю &
    Ошибка обнаружена, выводится сообщение \\
    \hline

    Совместимость с картами &
    Поддержка карт разных платёжных систем (МИР, Visa, Mastercard) &
    Устройство должно корректно работать со всеми совместимыми картами &
    Все тестовые карты успешно прошли инициализацию \\
    \hline

    Обновление прошивки &
    Процесс OTA-обновления микроконтроллера &
    Прошивка успешно загружена и применена, устройство продолжает работу &
    Обновление выполнено корректно \\
    \hline

    Восстановление после сбоя питания &
    Отключение питания на считывателе во время транзакции &
    Устройство должно корректно восстановиться, текущая транзакция отменяется &
    Устройство перезагружено, состояние системы очищено \\
    \hline
\end{longtable}

Тестирование интеграции частей системы проводилось в соответствии с описанием в подразделе~\ref{subsec:test_integr} ручным способом.
Для начала была протестирована реакция системы на поднесение бесконтактной платежный карты к NFC-модулю.
Если карта обнаруживалась и распознавалась корректно, то устройство выводило текст, информирующий об успешной идентификации карты и её уникальный идентификатор UID.
В случае возникновения ошибки на последовательный порт отправлялись сообщения с описанием проблемы, множественный вывод в различных местах позволял определять причину неисправности.

Для проверки работоспособности системы производились примитивные тесты.
Проверка корректной работы:
\begin{enumerate}
    \item поднесена карта с поддержкой стандарта устройства;
    \item ожидаемая реакция: успешное считывание и идентификация;
    \item полученный результат: успешное считывание.
\end{enumerate}

Проверка некорректного работы:
\begin{enumerate}
    \item поднесены неподдерживаемая карта;
    \item ожидаемая реакция: отсутствие реакции NFC-модуля;
    \item полученный результат: стандартный цикл использования устройства без взаимодействия с предметом/картой.
\end{enumerate}

На рисунке ~\ref{fig:test_success} показаны данные на Serial Monitor при успешной работе системы.

\begin{figure}[H]
    \centering
    \includegraphics[width=0.9\textwidth]{images/design/test_success}
    \caption{\centering Резльтат успешного тестирования системы}
    \label{fig:test_success}
\end{figure}
