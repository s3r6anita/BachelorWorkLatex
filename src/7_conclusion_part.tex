\newpage

\centeredsection{ЗАКЛЮЧЕНИЕ}
\addcontentsline{toc}{section}{ЗАКЛЮЧЕНИЕ}

В ходе выполнения выпускной квалификационной работы бакалавра была проведена полная разработка программно-аппаратной системы бесконтактной оплаты (СБО), ориентированной на интеграцию с российской платежной системамой <<МИР>>.
Был выполнен анализ современного состояния рынка мобильных платежей, изучены технологии реализации бесконтактной оплаты, включая стандарты EMV Contactless, ISO/IEC 14443, а также протоколы безопасности PCI PTS и PCI DSS.
На основе этого были установлены ключевые требования к аппаратной и программной частям системы.

Разработанная система состоит из двух основных компонентов:
\begin{itemize}
    \item аппаратной части — считывателя бесконтактных средств платежа (банковских карт и NFC-устройств);
    \item программной части — мобильного приложения для Android, обеспечивающего взаимодействие со считывателем и сервером банка через REST API.
\end{itemize}

Для аппаратной части были спроектированы электрические функциональная и принципиальные схемы, выбрана архитектура и компоненты и создан макет устройства-считывателя на основе требований стандартов безопасности и стандартов EMVCo.
Для программной части было создано мобильное приложение, реализующее логику обмена данными с терминалом и сервером банка-эквайера, с соблюдением требований стандарта PCI DSS, включая защиту передаваемых данных, аутентификацию запросов и логирование событий.

Кроме того, была реализована технология обмена данными по Bluetooth, что позволило создать гибкую и масштабируемую платформу для дальнейшего развития системы.
Программа учитывает особенности эксплуатации в условиях ограниченного бюджета и ориентирована на использование малым бизнесом и самозанятыми лицами.

На этапе тестирования проверены работоспособность, совместимость с бесконтактными картами ПС <<МИР>> и корректность интеграции с сервером банка.
Также была протестирована работа всех этапов транзакции: от инициализации до завершения операции с отображением результата пользователю.

Таким образом, была  реализована программно-аппаратная система бесконтактной оплаты, которая может стать экономически эффективной и доступной альтернативой существующим решениям.
Разработанное решение демонстрирует потенциал для дальнейшей коммерциализации и расширения функциональности, включая интеграцию с онлайн-кассами, поддержку токенизации и переход на формат смарт-терминалов.
