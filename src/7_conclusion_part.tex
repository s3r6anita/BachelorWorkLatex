\newpage

\centeredsection{ЗАКЛЮЧЕНИЕ}
\addcontentsline{toc}{section}{ЗАКЛЮЧЕНИЕ}

В ходе выполнения выпускной квалификационной работы бакалавра была проведена разработка программно-аппаратной системы бесконтактной оплаты (СБО), ориентированной на интеграцию с российской платежной системой <<МИР>>.
Был выполнен анализ современного состояния рынка бесконтактных платежей, изучены технологии их реализации, включая стандарты EMV Contactless, ISO/IEC 14443, а также протоколы безопасности PCI PTS и PCI DSS.
На основе этого были установлены ключевые требования к аппаратной и программной частям системы.

Были разработаны две основных части системы:
\begin{itemize}
    \item аппаратная~--- считыватель бесконтактных средств платежа (банковских карт и NFC-устройств с функцией оплаты);
    \item программная~--- мобильное приложения для Android, обеспечивающее взаимодействие со считывателем через Bluetooth и сервером банка через REST API.
\end{itemize}

Для аппаратной части были спроектированы электрические функциональная и принципиальные схемы, выбрана архитектура и компоненты, создан макет устройства-считывателя на основе требований стандартов безопасности и стандартов EMVCo.
Для программной части было создано мобильное приложение, реализующее логику обмена данными с терминалом и сервером банка-эквайера, с соблюдением требований стандарта PCI DSS, включая защиту передаваемых данных, аутентификацию запросов и логирование событий.

Кроме того, была реализована технология обмена данными по Bluetooth, что позволило создать гибкую и масштабируемую платформу для дальнейшего развития системы.
Программа учитывает особенности эксплуатации в условиях ограниченного бюджета и ориентирована на использование малым бизнесом и самозанятыми лицами.

Разработанное ПО было покрыто функциональными тестами для проверки корректности и идемпотентности их работы.
На этапе тестирования проверены работоспособность, совместимость с бесконтактными картами ПС <<МИР>> и корректность интеграции с сервером банка.

Таким образом, была  реализована программно-аппаратная система бесконтактной оплаты, которая может стать экономически доступной альтернативой существующим решениям или их дополнением.
Разработанное решение демонстрирует потенциал для дальнейшей коммерциализации и расширения функциональности, включая поддержку других платежных систем и контактного способа оплаты, а также интеграцию с онлайн-кассами и переход к формату смарт-терминала.
