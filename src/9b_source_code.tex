\newpage

\begin{center}
   \myAppendix{Фрагменты исходного кода}
   на 6 листах
\end{center}
\newpage

\ifthenelse{\boolean{website_upload}}{
  \includepdf{docs/source_title.pdf}
}{
  \ifthenelse{\boolean{test_vkr}}{
    \begin{figure}[H]
      \centering
      \includegraphics[height=0.95\textheight]{docs/title_source_code.png}
    \end{figure}
    \newpage
  }{
    \bmstutitleAppendix{Фрагменты исходного кода}
  }
}

\setcounter{lstlisting}{0}
\renewcommand{\thelstlisting}{\Asbuk{appendixNum}\arabic{lstlisting}}

\begin{singlespacing}
	\small
	\captionsetup{labelsep=endash, justification=raggedright, singlelinecheck=off}
	\lstinputlisting[language=c++, label=code:spi_init, linerange={1-50}, caption={Фрагмент настройки подключения PN5180 к МК}]{code/spi_init.cpp}
\end{singlespacing}



\pagebreak
\begin{singlespacing}
	\small
	\captionsetup{labelsep=endash, justification=raggedright, singlelinecheck=off}
	\lstinputlisting[language=c++, label=code:init, caption={Настройка подключения HC-05 к МК}]{code/init.cpp}
\end{singlespacing}



\begin{singlespacing}
    \small
    \captionsetup{labelsep=endash, justification=raggedright, singlelinecheck=off}
    \lstinputlisting[language=c++, label=code:control, linerange={1-22}, caption={Реализация управляющих методов для PN5180}]{code/control.cpp}
\end{singlespacing}
\pagebreak
{
    \noindent \small Продолжение листинга~\ref{code:control}
}
\vspace{-\baselineskip}
\begin{singlespacing}
    \small
    \captionsetup{labelsep=endash, justification=raggedright, singlelinecheck=off}
    \lstinputlisting[language=c++, linerange={23-64}, firstnumber=23, aboveskip=3mm]{code/control.cpp}
\end{singlespacing}
{
    \noindent \small Продолжение листинга~\ref{code:control}
}
\vspace{-\baselineskip}
\begin{singlespacing}
    \small
    \captionsetup{labelsep=endash, justification=raggedright, singlelinecheck=off}
    \lstinputlisting[language=c++, linerange={65-}, firstnumber=65, aboveskip=3mm]{code/control.cpp}
\end{singlespacing}



\begin{singlespacing}
	\small
	\captionsetup{labelsep=endash, justification=raggedright, singlelinecheck=off}
	\lstinputlisting[language=c++, label=code:read_serial, linerange={1-50}, caption={Чтение UID карты и проверки его корректности}]{code/read_serial.cpp}
\end{singlespacing}
{
    \noindent \small Продолжение листинга~\ref{code:read_serial}
}
\vspace{-\baselineskip}
\begin{singlespacing}
    \small
    \captionsetup{labelsep=endash, justification=raggedright, singlelinecheck=off}
    \lstinputlisting[language=c++, linerange={51-}, firstnumber=51, aboveskip=3mm]{code/read_serial.cpp}
\end{singlespacing}

% \addtocounter{page}{5}
