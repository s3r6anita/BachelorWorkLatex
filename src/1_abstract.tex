\newpage

\begin{center}
	\textbf{РЕФЕРАТ}
\end{center}

Записка на~\pageref{LastPage} с. (без учета приложений~\total{pagecount} с.), \total{figurescount} рис., \totaltables{} табл., \total{citenum} источников, \total{appendixcount} приложений.

БЕСКОНТАКТНАЯ ОПЛАТА, ПЛАТЕЖНЫЙ ТЕРМИНАЛ, МОБИЛЬНЫЕ ПЛАТЕЖИ, ЭКВАЙРИНГ, NFC, Android, STM32, SPI, USART.

Объектом разработки данной выпускной квалификационной работы бакалавра является программно-аппаратная система бесконтактной оплаты.

Цель работы~--- программно-аппаратная система, позволяющая интегрироваться с платежной системой банка и принимать оплату бесконтактными средствами платежа (банковскими картами и устройствами с поддержкой бесконтактной оплаты на территории Российской Федерации).

В процессе выполнения работы были выполнены следующие задачи: анализ технологий работы платежного терминала, проектирование и определение спецификаций аппаратной и программной частей системы, разработка и тестирование программного обеспечения системы, создание макета системы, тестирование системы и интеграции между ее частями.

В результате была спроектирована, разработана и протестирована программно-аппаратная система бесконтактной оплаты.
Пользователями данной системы являются банки и финансовые организации, предоставляющие услуги эквайринга.
