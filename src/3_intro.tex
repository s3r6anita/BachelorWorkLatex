\newpage

\centeredsection{ВВЕДЕНИЕ}
\addcontentsline{toc}{section}{ВВЕДЕНИЕ}

Бесконтактные платежи – это современный способ безналичной оплаты, который стал неотъемлемой частью повседневной жизни.
Их преимущество очевидно из названия~--- они не требуют физического контакта карты с терминалом, в результате чего ускоряется процесс выполнения оплаты и повышается скорость и качество обслуживания, особенно при покупках на небольшую сумму, когда нет необходимости вводить PIN-код карты.

Однако бесконтактный метод оплаты имеет свои нюансы: для его работы нужен платежный терминал, поддерживающий данную технологию, также при оплате на небольшие суммы карта не требует ввода пин-кода, вследствие чего любой человек, который завладеет ей, может воспользоваться ей для оплаты, также есть риск подмены платежного терминал мошенниками~\cite{codejournal}.
Несмотря на это бесконтактная оплата является наиболее распространенным способом оплаты как на глобальном, так и на российском рынке~\cite{posterminals}.
Согласно данным Банка России и Национальной системы платежных карт (НСПК), число операций с использованием бесконтактных технологий ежегодно увеличивается, и в 2024-м году доля безналичных платежей в розничном обороте составила 85,8\%.
Самым популярным средством безналичной оплаты остаются платежные карты~\cite{cdrf_report2024}\cite{cdrf_results2024}.
Что свидетельствует о высоком уровне доверия со стороны потребителей и ритейлеров и делает особенно актуальным разработку собственных решений бесконтактной оплаты, адаптированных под локальные требования и стандарты, а также обладающих гибкой архитектурой для дальнейшего масштабирования и интеграции.

На сегодняшний день существует множество коммерческих решений, предоставляющих возможность приема бесконтактных платежей.
Однако большинство из них ориентировано на крупные предприятия и требуют значительных финансовых и технических затрат при внедрении.
Таким образом, возникает потребность в создании компактной и экономически эффективной программно-аппаратной системы бесконтактной оплаты, которая может быть использована как малым бизнесом, так и частными лицами в условиях ограниченного бюджета.

Актуальность настоящей работы заключается в том, что современный рынок нуждается в доступных и простых в использовании решениях для организации бесконтактной оплаты, способных интегрироваться с существующими платежными системами и соответствовать всем необходимым требованиям безопасности.
Такие системы должны обеспечивать простоту установки, минимальное время настройки и устойчивость к внешним воздействиям в различных условиях эксплуатации.

Целью данной работы является проектирование и реализация программно-аппаратной системы бесконтактной оплаты (СБО), которая могла бы интегрироваться с платежной системой банка и позволяла принимать оплату бесконтактными средствами платежа (банковскими картами и устройствами с поддержкой бесконтактной оплаты на территории Российской Федерации).
СБО состоит из устройства для взаимодействия со средством платежа (считыватель бесконтактного средства платежа) и мобильного приложения для выполнения платежной транзакцией путем взаимодействия с платежным сервисом и получения данных для транзакции от средства платежа через устройство-считыватель.
